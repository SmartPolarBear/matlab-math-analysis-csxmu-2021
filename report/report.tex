\documentclass[12pt]{article}
\usepackage{amsmath}
\usepackage{mathtools}
\usepackage{graphics,epsfig}
\usepackage{xspace}
\usepackage{color}
\usepackage{amssymb}
\usepackage{subfigure}
\usepackage{multirow}
\usepackage{listings}
\usepackage{url}

\usepackage[noend]{algpseudocode}
\usepackage{algorithmicx,algorithm}

%%uncomment following line if you are going to use ``Chinese''
\usepackage{ctex} 

\definecolor{ballblue}{rgb}{0.13, 0.67, 0.8}
\definecolor{cornflowerblue}{rgb}{0.39,0.58,0.93}
\definecolor{babyblueeyes}{rgb}{0.63, 0.79, 0.95}

% preset-listing options
\lstset{
	basicstyle          =   \sffamily,          
	keywordstyle        =   \bfseries,         
	commentstyle        =   \rmfamily\itshape,  
	stringstyle         =   \ttfamily,  
	flexiblecolumns,               
	numbers             =   left, 
	showspaces          =   false, 
	numberstyle         =   \zihao{-5}\ttfamily,  
	showstringspaces    =   false,
	captionpos          =   t,    
	frame               =   lrtb,   
}

\lstdefinestyle{MatlabStyle}{
	language        =   Matlab, 
	basicstyle      =   \zihao{-5}\ttfamily,
	numberstyle     =   \zihao{-5}\ttfamily,
	keywordstyle    =   \color{blue},
	keywordstyle    =   [2] \color{teal},
	stringstyle     =   \color{magenta},
	commentstyle    =   \color{red}\ttfamily,
	breaklines      =   true, 
	columns         =   fixed,
	basewidth       =   0.5em,
}


\usepackage{geometry}
\geometry{
 a4paper,
 total={210mm,297mm},
 left=20mm,
 right=20mm,
 top=20mm,
 bottom=20mm,
 }

\marginparwidth = 10pt



\begin{document}
\title{Course Project Report: Advanced Math Analysis with Matlab}
\author{KeZheng Xiong \\ 22920202204622}

\maketitle

\abstract{The report for the end-of-term project of Advanced Math Analysis with MATLAB fall 2021 course. All the source code is open-sourced on the Github repository \url{https://github.com/SmartPolarBear/matlab-math-analysis-csxmu-2021} under \textbf{GPLv3} license}

\tableofcontents

\pagebreak

\section{Problem 1}

\subsection{Problem Description}
Given function $F(x, y) = 0.2x^2 + 0.1y^2 + sin(x + y)$, please work out its
gradient. Based on the gradient, please find out the local extreme of function F(x,y)
when both x and y are in the range of $[-2*\pi, 2*\pi]$. The 2D and 3D views of the function
is given in Fig. 1. 

\subsection{Solution}

\subsubsection{The gradient of the function}

I get the gradient of the function using the following code

\begin{lstlisting}[style=MatlabStyle,caption=Gradient Calculation]
	syms x y;
	f=0.2*x^2+0.1*y^2+sin(x+y);
	diff(f,x)
	diff(f,y)
\end{lstlisting}

Based on the result, the gradient is

\begin{equation}
	\nabla \cdot f(x,y)=( \frac{2*x}{5} + cos(x + y), \frac{y}{5} + cos(x + y))
\end{equation}

\subsubsection{Find the extreme values}
To find the extreme values of $F(x,y)$ with gradient decent method, we walk little steps towards the direction of the gradient. To formalize this idea, the algorithm is shown as follows.

\begin{algorithm}
	\caption{Gradient Descent}
	\hspace*{0.02in} {\bf Input:}
	Initial point $x_0$, a constant $\alpha$, $k=0$
	\begin{algorithmic}
	\While{termiation condition does not hold}
	\State $k=k+1$
	\State $x_{k+1}=x_k-\alpha \nabla \cdot f(x_k)$
	
	\EndWhile
	\end{algorithmic}
\end{algorithm}

Various problems occurs if this brute-force algorithm is implemented directly. The speed of convergence is annoyingly slow if parameters are not chosen right. In fact, I never succeeded finding a set of parameters that works. A well-known optimization is called Stochastic gradient descent, or SGD, improve it significantly.

The given parameter $\alpha$ in the brute-force algorithm, which is referred as learning rate, will change each round according to the situation. To be more exact, SGD tries to find a learning rate $m$, so that it minimize the function

\begin{equation}
	h(x,y,m) = \mathbf{x_0} + \nabla \cdot f(x,y)
\end{equation}

so the following equation is solved each round in the while-loop

\begin{equation}
	\frac{\partial h}{\partial m} = 0
\end{equation}

The implementation is shown in Code \ref{code:sgd}

\begin{lstlisting}[style=MatlabStyle,caption=SGD Gradient Descent,label=code:sgd]
	syms x y;
	f=0.2*x^2+0.1*y^2+sin(x+y);
	diff(f,x)
	diff(f,y)
\end{lstlisting}


\subsection{Acknowledgment}

Thanks to (TODO)


\end{document}
