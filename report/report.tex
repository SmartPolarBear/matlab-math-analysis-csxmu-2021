\documentclass[12pt]{article}
\usepackage{amsmath}
\usepackage{mathtools}
\usepackage{graphics,epsfig}
\usepackage{xspace}
\usepackage{color}
\usepackage{amssymb}
\usepackage{subfigure}
\usepackage{multirow}
\usepackage{listings}
\usepackage{url}
%%uncomment following line if you are going to use ``Chinese''
%\usepackage{ctex} 

\definecolor{ballblue}{rgb}{0.13, 0.67, 0.8}
\definecolor{cornflowerblue}{rgb}{0.39,0.58,0.93}
\definecolor{babyblueeyes}{rgb}{0.63, 0.79, 0.95}

% preset-listing options
\lstset{
  backgroundcolor=\color{white},   
  basicstyle=\footnotesize,    
  language=matlab,
  breakatwhitespace=false,         
  breaklines=true,                 % sets automatic line breaking
  captionpos=b,                    % sets the caption-position to bottom
  commentstyle=\color{ballblue},    % comment style
  extendedchars=true,              
  frame=single,                    % adds a frame around the code
  keepspaces=true,                 
  keywordstyle=\color{blue},       % keyword style
  numbers=left,                    
  numbersep=5pt,                   
  numberstyle=\tiny\color{blue}, 
  rulecolor=\color{babyblueeyes},
  stepnumber=1,              
  stringstyle=\color{black},     % string literal style
  tabsize=4,                       % sets default tabsize to 4 spaces
  title=\lstname                   
}

\usepackage{geometry}
\geometry{
 a4paper,
 total={210mm,297mm},
 left=20mm,
 right=20mm,
 top=20mm,
 bottom=20mm,
 }

\marginparwidth = 10pt



\begin{document}
\title{Course Project Report: Advanced Math Analysis with Matlab}
\author{KeZheng Xiong \\ 22920202204622}

\maketitle

\abstract{The report for the end-of-term project of Advanced Math Analysis with MATLAB fall 2021 course. All the source code is open-sourced on the Github repository \url{https://github.com/SmartPolarBear/matlab-math-analysis-csxmu-2021} under \textbf{GPLv3} license}

\tableofcontents

\pagebreak

\section{Problem 1}

\subsection{Problem Description}
Given function $F(x, y) = 0.2x^2 + 0.1y^2 + sin(x + y)$, please work out its
gradient. Based on the gradient, please find out the local extreme of function F(x,y)
when both x and y are in the range of $[-2*\pi, 2*\pi]$. The 2D and 3D views of the function
is given in Fig. 1. 

\subsection{Solution}

\subsubsection{The gradient of the function}

I get the gradient of the function using the following code

\begin{lstlisting}
	syms x y;
	f=0.2*x^2+0.1*y^2+sin(x+y);
	diff(f,x)
	diff(f,y)
\end{lstlisting}

Based on the result, the gradient is

\begin{equation}
	\nabla \cdot f(x,y)=( \frac{2*x}{5} + cos(x + y), \frac{y}{5} + cos(x + y))
\end{equation}

\subsubsection{Find the extreme values}
To find the extreme values of $F(x,y)$ with gradient decent method, we walk little steps towards the direction of the gradient. To formalize this idea, the algorithm is shown as follows.


\subsection{Acknowledgment}

Thanks to (TODO)


\end{document}
